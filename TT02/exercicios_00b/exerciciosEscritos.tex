\documentclass[12pt]{article}

\usepackage{listings}
\usepackage{graphicx,url}
\usepackage[utf8]{inputenc}
\usepackage[brazil]{babel}



\title{Trabalho Teórico 2\\Unidade 0}

\author{Iyan Lucas Duarte Marques}

\begin{document}

\maketitle

\section{Exercícios de revisão}
\subsection{Exercício p4}
O método converte o caractere recebido por parâmetro em um inteiro (utilizando a tabela ascii) e verifica
se o caractere em questão é uma vogal, maiúscula ou minúscula. 

\begin{scriptsize}


\begin{lstlisting}
    boolean doidao (char n){
        boolean resp= false;
        int v = (int) c;
        if (v == 65 || v == 69 || v == 73 || v == 79 || v == 85 || v == 97 || v == 101 || v ==105 ||
        v == 111 || v == 117){
            resp = true;
        }
    }

\end{lstlisting}
\end{scriptsize}
\subsection{Exercício p5}

\begin{scriptsize}
    
    \begin{lstlisting}
        boolean doidao (char n){
            boolean resp= false;
            int v = (int) c;
            if (v == 65 || v == 69 || v == 73 || v == 79 || v == 85 || v == 97 || v == 101 || v ==105 ||
            v == 111 || v == 117){
                resp = true;
            }
            return resp;
        }
        char toUpper(char c){
            return (c >= ‘a' && c <= ‘z') ? ((char) (c – 32)) : c ;
        }
        boolean isVogal (char c){
            c = toUpper(c);
            return (c =='A' || c =='E' || c =='I' || c =='O' || c =='U');
        }
    
    \end{lstlisting}
    \end{scriptsize}

    \subsection{Exercício p7}
O código simplificado:
\begin{scriptsize}

    
    \begin{lstlisting}
        import java.util.Scanner;

        public class temp {
            public static boolean isConsoante(String s, int n) {
                boolean resp = true;
                if (n != s.length()) {
                    if (s.charAt(n) < '0' || s.charAt(n) > '9') {
                        if (isVogal(s.charAt(n)) == true) {
                            resp = false;
                        } else {
                            resp = isConsoante(s, n + 1);
                        }
                    } else {
                        resp = false;
                    }
                }
                return resp;
            }
        
            public static void main(String[] args) {
                Scanner input = new Scanner(System.in);
                String s = input.nextLine();
                System.out.println(isConsoante(s, 0));
            }
        }
    \end{lstlisting}
    \end{scriptsize}
    \subsection{Exercício p16}
    A primeira versão é mais fácil de entender pelo fato de estar mais simplificada e concisa.
    \subsection{Exercício p17}
    A minha opinião é que o código é confuso e redundante 
    \subsection{Exercício p18}
    A diferença entre os códigos é que, no m1, i é decrescido depois da operação se realizar. Em m2,
    i é decrescido antes de realizar a operação de return.
    \begin{scriptsize}
        
        \begin{lstlisting}
            int m1(int i){
                return i--;
            }
                
            int m2(int i){
                return --i;
            }
        \end{lstlisting}
        \end{scriptsize}
        \subsection{Exercício p19}
        O  programa mostra vários tipos de dados, entre eles:
        \begin{itemize}
            \item byte: consume 1 byte, range (valor máximo e mínimo): 127
            \item short: consume 2 bytes, range: 32.767
            \item int: consume 4 bytes, range: 2,14*10a9
            \item long: consume 8 bytes, range: 9.23*10a18
        \end{itemize}
        \begin{scriptsize}
            
            \begin{lstlisting}
                byte b = 0; short s = 0; int i = 0; long l = 0;
                while (true){
                    b++; s++; i++; l++;
                    System.out.println(b + “ ” + s + “ ” + i + “ ” + l);
                }
            \end{lstlisting}
            \end{scriptsize}

            \subsection{Exercício p20}
            O  programa imprime [46-11] pelo fato de ao usar o operador \textit{shift}, os bits são 
            deslocados para a direção das aspas francesas pelo número de posições especificadas pela 
            expressão aditiva. As posições de bits que foram liberadas pela operação de deslocamento 
            são preenchidas com zeros. Dessa forma, o número 23 que em binário é 10111 foi deslocado 
            uma casa para a esquerda resultando no número 101110, que é 46. O número 23 foi deslocado
            para a direita, o que resultou no número 11, que é 1011 em binário.
            \begin{scriptsize}

                
                \begin{lstlisting}
                    int x = 23, y = 23;
                    x = x << 1;
                    y = y >> 1;
                    System.out.println(“[” + x + “ - ” + y + “]”);
                \end{lstlisting}
                \end{scriptsize}
                \section{Exercícios Resolvidos}
                \subsection{Exercício p4}
                O código verifica se o caractere é o mesmo que as letras cujos números da tabela ascii
                correspondem aos no if, ou seja, uma vogal.
                \subsection{Exercício p14}
                O primeiro passo ao corrigir o código foi a indentação, após isso, foi simplificado
                o if else.


\end{document}
