\documentclass[12pt]{article}

\usepackage{listings}
\usepackage{graphicx,url}
\usepackage[utf8]{inputenc}
\usepackage[brazil]{babel}

\title{Handbook de Somatórios}
\author{Iyan Lucas Duarte Marques}
\begin{document}
\maketitle
\section{Siglas e abreviações}
\begin{itemize}
\item $S_n$ somatório (ou soma) até o número natural n
\item $a_i$ ou $b_i$ é o termo correspondente ao índice i na posição descrita, usado dessa forma 
para simbolizar um número/constante genérica
\item As seguintes notações significam a mesma coisa:
    \begingroup
    \LARGE
    \begin{equation}
        \sum_{0}^{n}i == \sum_{0\leq i\leq n}
    \end{equation}
    \endgroup

\end{itemize}


\section{Fórmulas e propriedades}
\subsection{Distributividade}
Permite mover constantes para dentro ou para fora de um somatório:

\begingroup
\LARGE
\begin{equation}
\sum_{i\in I}(c*a_i) = c*\sum_{i\in I}a_i
\end{equation}
\endgroup
\subsection{Associatividade}
Permite quebrar um somatório em duas partes ou combinar dois Somatórios em um:

\begingroup
\LARGE
\begin{equation}
\sum_{i\in I}(a_i+b_i) = \sum_{i\in I}a_i + \sum_{i\in I}b_i
\end{equation}
\endgroup

\subsection{Comutatividade}
Permite colocar os termos em qualquer ordem:
\begingroup
\LARGE
\begin{equation}
\sum_{i\in I}a_i = \sum_{p(i)\in I}a_{p(i)}
\end{equation}
\endgroup

\subsection{Progressão Aritmética (PA)}
A Progressão Aritmética (P.A.) é uma sequência de números onde a diferença entre dois termos consecutivos é sempre a mesma. Essa diferença constante é chamada de razão da P.A..

Sendo assim, a partir do segundo elemento da sequência, os números que surgem são resultantes da soma da constante com o valor do elemento anterior.

\subsubsection{Termo Geral}
$a_n = a_1+(n-1)*r$

$a_n$ : termo que queremos calcular

$a_1$: primeiro termo da P.A.

n: posição do termo que queremos descobrir

r: razão
\subsubsection{Soma de uma PA}
\begin{itemize}
    \item Equação "normal": 
        \begingroup
        \LARGE
        \begin{equation}
            S_n = \frac{(a_1+a_n)*n}{2}
        \end{equation}
        \endgroup
        $S_n$: soma dos n primeiros termos da P.A.

        $a_1$: primeiro termo da P.A.
        
        $a_n$: ocupa a enésima posição na sequência
        
        $n$: posição do termo
    \item somatório: 
        \begingroup
        \LARGE
        \begin{equation}
            S_n = \sum_{0\leq i\leq n}[a+b*i]
        \end{equation}
        \endgroup
        $S_n$: soma dos n primeiros termos da P.A.

        $a$: termo inicial

        $b$: razão

        $i$: posição do índice

        $n$: limite do índice
    \item Equação baseada no somatório:
        \begingroup 
        \LARGE
        \begin{equation}
            S_n = \frac{(2a+bn)(n+1)}{2}
        \end{equation}
        \endgroup
        A legenda é a mesma do somatório.
\end{itemize}
\subsection{Conjuntos Numéricos (P1)}
Combina Somatórios de índice diferentes. No caso se os conjuntos A e B são dois
conjuntos quaisquer de inteiros:

\begingroup 
\LARGE
\begin{equation}
\sum_{i\in A}a_i + \sum_{i\in B}a_i = \sum_{i\in A\cap B}a_i + \sum_{i\in A\cup B}a_i
\end{equation}
\endgroup
Se $A = {1,2,3}$ e $B = {3,5,7}$, então, $A\cup B = {1,2,3,4,5,7}$ e $A\cap B = {3}$
\par Observe que a união garante todos os elementos e a interseção, os repetidos.

\subsection{Pertubação (P2)}
Dada uma soma genérica qualquer $S_n = \sum_{0\leq i\leq n}a_i$
podemos reescrever $S_{n+1 = a_0 +a_1 +a_2 +\dots+ a_{n+1}}$
de duas formas:
\begin{enumerate}
    \item Forma:
        \begingroup 
        \LARGE
        \begin{equation}
            S_{n+1}=S_n + a_{n+1}
        \end{equation}
        \endgroup
    \item Forma
        \begingroup 
        \LARGE
        \begin{equation}
            S_{n+1} = a_0 + \sum_{0\leq i\leq n}a_{i+1}
        \end{equation}
        \endgroup
\end{enumerate}
\par Na prática, para perturbar o somatório, resolvemos a seguinte igualdade abaixo:
\begingroup 
\LARGE
\begin{equation}
    S_{n} + a_{n+1} = a_0 + \sum_{0\leq i\leq n}a_{i+1}
\end{equation}
\endgroup
(A junção das duas equações acima).
Isso, frequentemente, resulta na equação fechada para $S_n$.
\subsection{$i*x^i$}
Quando um somatório tiver o índice i multiplicado por uma constante
elevada ao índice, teremos esta fórmula:

\begingroup
\LARGE
\begin{equation}
    S_n = \sum_{0\leq i\leq n}i*x^i = (n-1)*x^{n+1} + x
\end{equation}
\endgroup
\subsection{somatório de 1}
\begingroup 
\LARGE
\begin{equation}
\sum_{0\leq i \leq n}1
\end{equation}
\endgroup
O somatório de 1 é sempre $(n+1)$
\subsection{$a_i$}
Seja um somatório:
\begingroup 
\LARGE
\begin{equation}
S_n = \sum_{0\leq i\leq n}a*x^i
\end{equation}
\endgroup
Para descobrir a constante multiplicada (a), faz-se a seguinte fórmula:
\begingroup 
\LARGE
\begin{equation}
a_i = a*x^i
\end{equation}
\endgroup
\subsection{Somas Múltiplas}
Outra forma de representação é utilizando dois Somatórios, por exemplo:
\begingroup 
\LARGE
\begin{equation}
\sum_{1\leq i,j\leq 3}a_ib_j = (\sum_{1\leq i\leq 3}a_i)(\sum_{1\leq j\leq 3}b_j)
\end{equation}
\endgroup
\subsection{Somatório de i}
\begingroup 
\LARGE
\begin{equation}
\sum_0^ni=\frac{n(n+1)}{2}
\end{equation}
\endgroup
\section{Prova por Indução}
\begin{enumerate}
    \item Passo (passo base): Provar que a fórmula é verdadeira para o primeiro valor
    (na equação, substitua n pelo primeiro valor).
    \item Passo (Indução propriamente dita): Supondo que $n>0$ e que a fórmula é verdadeira quando trocamos n por (n-1).
    \begingroup 
\LARGE
\begin{equation}
S_n = S_{n-1} + a_n
\end{equation}
\endgroup
\end{enumerate}

\end{document}